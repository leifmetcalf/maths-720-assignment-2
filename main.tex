\documentclass[a4paper, 12pt]{article}
\usepackage[utf8]{inputenc}
\usepackage{mathtools}
\usepackage{amssymb}
\usepackage{enumitem}
\usepackage{parskip}
\usepackage{xfrac}
\usepackage{xcolor}
\usepackage{bm}
\usepackage{booktabs}

\newcommand{\tr}{^{\mathsf{T}}}
\newcommand{\N}{\mathbb{N}}
\newcommand{\R}{\mathbb{R}}
\newcommand{\Z}{\mathbb{Z}}
\newcommand{\Q}{\mathbb{Q}}
\newcommand{\half}{\sfrac{1\!}2}
\DeclarePairedDelimiter\abs{\lvert}{\rvert}
\DeclarePairedDelimiter\inner{\langle}{\rangle}
\DeclareMathOperator{\GL}{GL}
\DeclareMathOperator{\reduce}{reduce}
\DeclareMathOperator{\im}{Im}

\setlist[enumerate, 1]{leftmargin=0pt, label=\textbf{\arabic*.}}

\begin{document}

\begin{enumerate}

\item Let \(G\) be a finite group in which every two non-identity elements are conjugate. Clearly if \(G\) is trivial then \(\abs{G}\leq2\), so suppose \(G\) is non-trivial and let \(a\) be a non-identity element of \(G\). Then since every two non-identity elements are conjugate the size of the conjugacy class of \(a\) is \(\abs{G}-1\). By the orbit--stabiliser theorem the size of this conjugacy class divides the order of \(G\), but the only way for \(\abs{G}-1\) to divide \(\abs{G}\) is if \(\abs{G}=2\). Hence \(\abs{G}\leq2\).

\item Let \(G\) be a finite group that acts transitively on a set \(\Omega\) of size \(n\geq2\). Since \(G\) is transitive there is only one orbit so by the Cauchy--Frobenius lemma the average number of fixed points of elements in \(G\) is 1. The identity element of \(G\) has \(\abs\Omega\geq2\) fixed points, so for the average number of fixed points to be 1 there must at least one element of \(G\) with 0 fixed points.

\item \begin{enumerate}

\item Let \(G\) be a finite group and let \(p\) be the smallest prime divisor of \(\abs{G}\). Let \(H\) be a subgroup of index \(p\) in \(G\). Let \(\phi\colon G\to S_p\) be the action of \(G\) by left multiplication on left cosets of \(H\) in \(G\). By the first isomorphism theorem
\[\frac{G}{\ker\phi}\cong\im\phi\leq S_p\]
so \(\abs{G:\ker\phi}\) divides \(\abs{S_p}=p!\). Also, \(\frac G{\ker\phi}\leq G\) so the smallest prime divisor of \(\abs{G:\ker\phi}\) is at least \(p\). Note \(\abs{G:\ker\phi}\neq1\)  since \(\ker\phi\leq H\). Hence \(\abs{G:\ker\phi}=\abs{G:H}=p\) and so \(\abs{\ker\phi}=\abs{H}\).

Since \(\ker\phi\leq H\) and \(\abs{\ker\phi}=\abs{H}\), it follows that \(H=\ker\phi\) and so \(H\) is normal.

\item Let \(G\) be a simple group of order 1092. Suppose \(G\) has a subgroup of order 182. Let \(\phi\colon G\to S_p\) be the action of \(G\) by left multiplication on left cosets of \(H\) in \(G\). Note since \(G\) is simple \(\ker\phi\) is trivial. By the first isomorphism theorem
\[\frac{G}{\ker\phi}\cong\im\phi\leq S_6.\]
But \(\ker\phi\) is trivial so \(G\) is isomorphic to a subgroup of \(S_6\). But \(13\) is a prime divisor of the order of \(G\) which doesn't divide the order of \(S_6\), a contradiction. Hence \(G\) has no subgroup of order 182.

\end{enumerate}

\item Let \(G\) be a finite group of order \(p^2q\) where \(p\) and \(q\) are distinct primes. By \(3(a)\) and Sylow theory we can assume \(p<q\). By the first Sylow theorem there are either \(1\), \(p\), or \(p^2\) conjugate Sylow \(q\)-subgroups. In the first case the unique Sylow \(q\)-subgroup is normal. By the third Sylow theorem the number of conjugates is equivalent to 1 modulo \(q\), and since \(p<q\) we can rule out the case with \(p\) conjugates. Suppose there are \(p^2\) conjugate Sylow \(q\)-subgroups. Since \(q\) is prime these intersect trivially and so there are \(p^2(q-1)\) elements of order \(q\) in \(G\). This leaves a maximum of \(p^2\) elements of order not equal to \(q\) and so there is only one Sylow \(p^2\)-subgroup which is hence normal. Thus there always exists a normal subgroup and so \(G\) is not simple.

\item Let \(G\) be a finite group and let \(N\) be a normal subgroup of \(G\). Let \(P\) be a Sylow \(p\)-subgroup of \(G\). Note since \(N\) is normal \(PN\) is a subgroup of \(G\). Note also by Lagrange's theorem the order of \(P\cap N\) is a power of \(p\). Since
\[\abs{PN}=\frac{\abs P\abs N}{\abs{P\cap N}}\]
we have
\[\abs{PN:P}=\abs{N:P\cap N}.\]
Then since \(\abs{G:P}\) is coprime to \(p\) and \(\abs{G:P}=\abs{G:PN}\abs{PN:P}\) so too is \(\abs{N:P\cap N}=\abs{PN:P}\) coprime to \(p\). Thus \(\abs{P\cap N}\) is a power of \(p\) and \(\abs{N:P\cap N}\) is coprime to \(p\), so \(P\cap N\) is a Sylow \(p\)-subgroup of \(N\).

\item Let \(G\) be a group with presentation \(\langle\,x,y\mid x^{-1}yxy^{-2},\,y^{-1}xyx^{-2}\,\rangle\). We have
\begin{align*}
&x^{-1}yxy^{-2}=1\\
\implies\quad&x^{-1}y(y^{-1}xyx^{-2})xy^{-2}=1\\
\implies\quad&yx^{-1}y^{-2}=1\\
\implies\quad&x^{-1}=y
\end{align*}
which plugged into the original identities gives \(x=1\) and \(y=1\), so \(G\) is trivial.

\item Let \(F_n=\langle\,x_1,x_2,\dots,x_n\mid\,\,\rangle\) be the free group of rank \(n\) where \(n\geq2\).

\begin{enumerate}

\item Since \(n\geq2\) the free group generated by \(x_1\) and \(x_2\) is a subgroup of \(F_n\) of rank 2 and hence isomorphic to \(F_2\).

\item Let \(k\geq2\). Consider the homomorphism \(\phi:F_k\to F_2\) that maps \(y_i\) to \(x_2^{-i}x_1x_2^i\):
\[y_{\lambda_1}^{\varepsilon_1}\dots y_{\lambda_n}^{\varepsilon_n} \mapsto \reduce{x_2^{-\lambda_1}x_1^{\varepsilon_1}x_2^{\lambda_1}\dots x_2^{-\lambda_n}x_1^{\varepsilon_n}x_2^{\lambda_n}}\]
The only way there can be a non-trivial kernel is if there is some \(y_{\lambda_1}^{\varepsilon_1}\dots y_{\lambda_n}^{\varepsilon_n}\in F_k\) for which some two adjacent \(\lambda_i\) are equal. But the elements of \(F_k\) are reduced words so this is impossible. Hence \(\ker\phi\) is trivial and so by the first isomorphism theorem \(G\cong\im\phi\leq F_2\).

\item By a similar argument to 7(a) there exists a subgroup of \(F_n\) isomorphic to \(F_1\). For \(k\geq2\), 7(b) gives a subgroup of \(F_2\) isomorphic to \(F_k\) which combined with 7(a) gives a subgroup of \(F_n\) isomorphic to \(F_k\).

\end{enumerate}

\item \begin{enumerate}

\item Let \(G\) be a group generated by two elements \(x\) and \(y\) of orders 2 and 3 such that \(xy\) has order 6. Let \(N\) be the subgroup generated by the two elements \(u\coloneqq xy^{-1}xy\) and \(v\coloneqq xyxy^{-1}\).
\begin{align*}
u^{-1}&=y^{-1}xyx\\
v^{-1}&=yxy^{-1}x\\
xux&=u^{-1}\\
y^{-1}uy&=u^{-1}v\\
yuy^{-1}&=v^{-1}\\
xvx&=v^{-1}\\
y^{-1}vy&=u^{-1}\\
yvy^{-1}&=v^{-1}u
\end{align*}
And since \(g^{-1}n^{-1}g=(g^{-1}ng)^{-1}\) for all \(g\in G\) and \(n\in N\), \(N\) is closed under conjugation by \(x\), \(y\) and \(y^{-1}\) in \(G\). Hence \(N\) is normal in \(G\).

\item Note
\begin{align*}
[u,v]&=u^{-1}v^{-1}uv\\
&=y^{-1}xyxyxy^{-1}xxy^{-1}xyxyxy^{-1}\\
&=y^{-1}xyxyxyxyxyxy^{-1}\\
&=y^{-1}(xy)^6y\\
&=1
\end{align*}
so \(u\) and \(v\) commute and so \(N\) is Abelian.

\item Note
\begin{gather*}
vyx=xy\\
uy^{-1}x=xy^{-1}\\
v^{-1}xyx=y\\
u^{-1}xy^{-1}x=y^{-1}
\end{gather*}
so the right-coset enumeration is
\begin{center}
\begin{tabular}{@{}llll@{}} \toprule
&\(x\)&\(y\)&\(y^{-1}\)\\ \midrule
\(1=N\)&2&3&4\\
\(2=Nx\)&1&5&6\\
\(3=Ny\)&5&4&3\\
\(4=Ny^{-1}\)&6&1&3\\
\(5=Nxy\)&3&6&2\\
\(6=Nxy^{-1}\)&4&2&3\\ \bottomrule
\end{tabular}
\end{center}
Hence \(\abs{G:N}\) is at most six and so by Lagrange's theorem is either 1, 2, 3 or 6.

\end{enumerate}

\item

\end{enumerate}

\end{document}